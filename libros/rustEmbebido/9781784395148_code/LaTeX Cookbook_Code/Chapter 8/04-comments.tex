\documentclass[paper=a4,oneside,fontsize=12pt,
  parskip=full]{scrartcl}
\usepackage[svgnames]{xcolor}
\usepackage[author={Stefan Kottwitz},icon=Note,color=Yellow,open=true,final]{pdfcomment}
\begin{document}
%\tableofcontents
\addsec{Introduction}
This document will be our starting point for simple
documents. It is suitable for a single page or up to
a couple of dozen pages.
\pdfcomment{Simple documents don't have chapters.}

The text will be divided into \pdfmarkupcomment{sections}{You could additionally use subsections.}.
\section{The first section}
This first text will contain
\begin{pdfsidelinecomment}[color=Red]{A bulleted list}
\begin{itemize}
\item a table of contents,
\item a bulleted list,
\item headings and some text and math in section,
\item referencing such as to section \ref{sec:maths} and
      equation (\ref{eq:integral}).
\end{itemize}
\end{pdfsidelinecomment}
We can use this document as a template for filling in
our own content.
\section{Some maths}
\label{sec:maths}
When we write a scientific or technical document, we usually
include math \pdftooltip{formulas}{Formulas can be inline or displayed in their own paragraph}. To get a brief glimpse of the look of
maths, we will look at an integral approximation of a function
$f(x)$ as a sum with weights $w_i$:
\begin{equation}
  \pdfmargincomment{The equation environment produces a centered
  equation with whitespace before and after it.}
  \label{eq:integral}
  \int_a^b \pdfmarkupcomment{f(x)}{This function will be approximated.}\,\mathrm{d}x \approx (b-a)
  \sum_{i=0}^n \pdfmarkupcomment{w_i}{Those are the weights} f(x_i)
\end{equation}
\pdffreetextcomment[subject={Summary},width=7.5cm,height=2.2cm,opacity=0.5,voffset=-3cm]{The whole document is an example
showing how to write a small document.
Now we enriched it with sample comments.}
\end{document}
